%!TEX TS-program = xelatex
%!TEX encoding = UTF-8 Unicode
% Awesome CV LaTeX Template for Cover Letter
%
% This template has been downloaded from:
% https://github.com/posquit0/Awesome-CV
%
% Authors:
% Claud D. Park <posquit0.bj@gmail.com>
% Lars Richter <mail@ayeks.de>
%
% Template license:
% CC BY-SA 4.0 (https://creativecommons.org/licenses/by-sa/4.0/)
%


%-------------------------------------------------------------------------------
% CONFIGURATIONS
%-------------------------------------------------------------------------------
% A4 paper size by default, use 'letterpaper' for US letter
\documentclass[11pt, a4paper]{awesome-cv}

% Configure page margins with geometry
\geometry{left=1.4cm, top=.8cm, right=1.4cm, bottom=1.8cm, footskip=.5cm}

% Specify the location of the included fonts
\fontdir[fonts/]

% Color for highlights
% Awesome Colors: awesome-emerald, awesome-skyblue, awesome-red, awesome-pink, awesome-orange
%                 awesome-nephritis, awesome-concrete, awesome-darknight
\colorlet{awesome}{awesome-red}
% Uncomment if you would like to specify your own color
% \definecolor{awesome}{HTML}{CA63A8}

% Colors for text
% Uncomment if you would like to specify your own color
% \definecolor{darktext}{HTML}{414141}
% \definecolor{text}{HTML}{333333}
% \definecolor{graytext}{HTML}{5D5D5D}
% \definecolor{lighttext}{HTML}{999999}

% Set false if you don't want to highlight section with awesome color
\setbool{acvSectionColorHighlight}{true}

% If you would like to change the social information separator from a pipe (|) to something else
\renewcommand{\acvHeaderSocialSep}{\quad\textbar\quad}


%-------------------------------------------------------------------------------
%	PERSONAL INFORMATION
%	Comment any of the lines below if they are not required
%-------------------------------------------------------------------------------
% Available options: circle|rectangle,edge/noedge,left/right
\photo[circle,noedge,left]{./picture.jpg}
\name{Andrés}{Cabero Busto}
\position{Machine Learning Engineer{\enskip\cdotp\enskip}Software Engineer}
\address{Allmandring 18C-15, Stuttgart, Germany}

\mobile{+4917684048732}
\email{cabero96@protonmail.com}
\github{cbr9}
\linkedin{andres-cabero-busto}



%-------------------------------------------------------------------------------
%	LETTER INFORMATION
%	All of the below lines must be filled out
%-------------------------------------------------------------------------------
% The company being applied to
\recipient
  {Company Recruitment Team at Volue}
  {} % Address
% The date on the letter, default is the date of compilation
\letterdate{\today}
% The title of the letter
\lettertitle{Job Application for Quantitative Developer}
% How the letter is opened
\letteropening{To whom it may concern,}
% How the letter is closed
\letterclosing{Kind regards,}
% Any enclosures with the letter
% \letterenclosure[Attached]{Curriculum Vitae}


%-------------------------------------------------------------------------------
\begin{document}

% Print the header with above personal informations
% Give optional argument to change alignment(C: center, L: left, R: right)
\makecvheader[R]

% Print the footer with 3 arguments(<left>, <center>, <right>)
% Leave any of these blank if they are not needed
\makecvfooter
  {\today}
  {Andrés Cabero Busto~~~·~~~Cover Letter}
  {}

% Print the title with above letter informations
\makelettertitle

%-------------------------------------------------------------------------------
%	LETTER CONTENT
%-------------------------------------------------------------------------------
\begin{cvletter}

\lettersection{Quick Note On My Technical Qualifications}
I am a motivated and passionate master's student specializing in Natural Language Processing (NLP) and Automatic Speech Recognition (ASR). I have a solid academic background and plenty of hands-on experience. During the course of my studies, I have developed a deep understanding of machine learning techniques and frameworks. 

Since I started my journey in NLP/ML in 2018, I have worked with, and mastered, frameworks such as NLTK, Spacy, HuggingFace, PyTorch, PyTorch Lightning, Captum (for explainable AI) and Weights \& Biases. Most recently, during my Master's thesis, I have trained models using multiple GPUs using the Slurm Workload Manager.

I am also an experienced and passionate Rust developer and Linux user. Check out one of my side projects at \url{https://github.com/cbr9/organizer}, a CLI tool to manage files via user-defined rules, as well as my NixOS configuration (\url{https://github.com/cbr9/dotfiles}), which I use to maintain the exact same workflow across all of my machines.

% \lettersection{Why Google?}
% Suspendisse commodo, massa eu congue tincidunt, elit mauris pellentesque orci, cursus tempor odio nisl euismod augue. Aliquam adipiscing nibh ut odio sodales et pulvinar tortor laoreet. Mauris a accumsan ligula. Class aptent taciti sociosqu ad litora torquent per conubia nostra, per inceptos himenaeos. Suspendisse vulputate sem vehicula ipsum varius nec tempus dui dapibus. Phasellus et est urna, ut auctor erat. Sed tincidunt odio id odio aliquam mattis. Donec sapien nulla, feugiat eget adipiscing sit amet, lacinia ut dolor. Phasellus tincidunt, leo a fringilla consectetur, felis diam aliquam urna, vitae aliquet lectus orci nec velit. Vivamus dapibus varius blandit.

\lettersection{Why Me?}
In addition to my technical skills, which can be gauged from the attached CV and the previous section of this letter, I'd like to tell you about how I approach things and difficulties.

I believe in the structured approach to work. As such, I bring strong productivity and time-management skills, such as time-blocking. I also believe in collaboration and direct communication between teammates (this is one of the reason why I prefer on-site work rather than working from home).

I also believe that if I don't know something, I can learn it, no matter what it is. For example, as you can see from my resumé, I did my Bachelor's degree in Linguistics, where I was introduced to programming and NLP. Up until this point I hadn't studied math since 3-4 years before, so my math skills had stagnated and there was no one to teach me. I also knew that I wanted to pursue a career in AI. This led me to teach myself mathematics, starting from probability and statistics and ending with vector calculus, using a range of different online resources. Since then I have continued to apply this to my daily learning, be it with (human) languages, technologies, or anything in between.

Overall, I would argue that whereas I don't have multiple years of experience (yet), I compensate for it with my soft skills and structured approach.

\end{cvletter}


%-------------------------------------------------------------------------------
% Print the signature and enclosures with above letter informations
\makeletterclosing

\end{document}
