%-------------------------------------------------------------------------------
%	SECTION TITLE
%-------------------------------------------------------------------------------
\cvsection{Experience}


%-------------------------------------------------------------------------------
%	CONTENT
%-------------------------------------------------------------------------------
\begin{cventries}

%---------------------------------------------------------
  \cventry
    {Master Thesis}
    {Sony Europe B.V. (AI, Speech \& Sound Group)} % Organization
    {Stuttgart, Germany} % Location
    {Apr. 2023 - Oct. 2023} % Date(s)
    {
      \begin{cvitems} % Description(s) of tasks/responsibilities
        \item Topic: Multi-domain adaptation of self-supervised speech transformers via continual curriculum learning % Job title
        \item Performed domain-adaptation on an open-source, self-supervised (SSL) speech recognition model.
        \item Applied regularization-based and replay-based continual learning methods to avoid incurring into catastrophic forgetting on the old Librispeech ASR task.
      \end{cvitems}
    }

  \cventry
    {Working Student} % Job title
    {Sony Europe B.V. (AI, Speech \& Sound Group)} % Organization
    {Stuttgart, Germany} % Location
    {Jul. 2022 - Mar. 2023} % Date(s)
    {
      \begin{cvitems} % Description(s) of tasks/responsibilities
        \item Developed a natural language grammar based on weighted finite-state transducers (WFSTs) and the Thrax framework, similar to \href{https://github.com/NVIDIA/NeMo-text-processing}{NVIDIA NeMo grammars}, that enabled a 10\% WER improvement for the pre-existing Italian ASR system.
        \item Time-aligned and transcribed speech utterances using \href{https://labelstud.io/}{Label Studio} for use in text-to-speech (TTS) model training.
      \end{cvitems}
    }

  \cventry
    {Research Assistant} % Job title
    {Institut für Maschinelle Sprachverarbeitung (IMS)} % Organization
    {Stuttgart, Germany} % Location
    {Jul. 2022 - Dec. 2022} % Date(s)
    {
      \begin{cvitems} % Description(s) of tasks/responsibilities
        \item Developed a toolkit for natural language processing research in lexical semantic change detection (LSCD)
        \item Set up the project using Hydra for easy experimentation and customizability
        \item Data validation done using Pydantic
      \end{cvitems}
    }

  \cventry
    {Linux System Administrator} % Job title
    {Institut für Maschinelle Sprachverarbeitung (IMS)} % Organization
    {Stuttgart, Germany} % Location
    {Jul. 2022 - Mar. 2022} % Date(s)
    {
      \begin{cvitems} % Description(s) of tasks/responsibilities
        \item Advising students and employees at IMS' IT management office
        \item Management and installation of personal computers and GPU servers
        \item Management of users, network \& websites
        \item Technical support and assistance for professors and students
      \end{cvitems}
    }

  \cventry
    {Data Annotator} % Job title
    {Institut für Maschinelle Sprachverarbeitung (IMS)} % Organization
    {Stuttgart, Germany} % Location
    {Nov. 2021 - Jan. 2022} % Date(s)
    {
      \begin{cvitems} % Description(s) of tasks/responsibilities
        \item Data annotator for the first Shared Task of Lexical Semantic Change Discovery in Spanish (\url{https://fdzr.github.io/lscdiscovery/}). 
        \item The process involved annotating the degree of difference between uses of the same word based on the surrounding context, using a Likert scale.
		\item Identified practical difficulties in the annotation platform, and developed a JavaScript solution to enhance the efficiency for the project and the team.
      \end{cvitems}
    }


%---------------------------------------------------------
\end{cventries}
